\section{Задание}

Написать программу на языке Java, выполняющую соответствующие варианту действия. Программа должна соответствовать следующим требованиям:
\begin{enumerate} 
    \item Она должна быть упакована в исполняемый jar-архив.
    \item Выражение должно вычисляться в соответствии с правилами вычисления математических выражений (должен соблюдаться порядок выполнения действий и т.д.).
    \item Программа должна использовать математические функции из стандартной библиотеки Java.
    \item Результат вычисления выражения должен быть выведен в стандартный поток вывода в заданном формате.
\end{enumerate}
Выполнение программы необходимо продемонстрировать на сервере helios.\newline \newline

\begin{enumerate} 
    \item Создать одномерный массив a типа short. Заполнить его числами от 3 до 20 включительно в порядке возрастания.
    \item Создать одномерный массив x типа double. Заполнить его 10-ю случайными числами в диапазоне от -13.0 до 4.0.
    \item Создать двумерный массив a размером 18x10. Вычислить его элементы по следующей формуле (где x = x[j]): 
        \begin{itemize}
            \item если a[i] = 7, то a[i][j]= e$^{\cos (\cos (x))}$ 
            \item если $a[i] \in \{4, 5, 9, 10, 12, 14, 17, 19, 20\}$, то a[i][j]= $\arcsin (\frac{1}{e^{\tan^2 (e^{e^x}) } })$ 
            \item для остальных значений $a[i]: a[i][j]=1 - (\frac{1}{4} \cdot e^{\tan(x)})^{2}$
        \end{itemize}
    \item Напечатать полученный в результате массив в формате с тремя знаками после запятой.        
\end{enumerate}
